\documentclass[a4paper, 11pt, toc=listof, toc=bib]{scrartcl}

% Hier Daten eintragen:
\newcommand{\name}{{{Ihr Name}}}
\newcommand{\email}{{{E-Mail}}}
\newcommand{\matnr}{{{Matrikelnummer}}}
\newcommand{\thema}{{{Titel des Projekts}}}
\newcommand{\datum}{\today}
% Ende der Daten

\usepackage[left=2cm, right=2cm, top=2cm, bottom=2cm]{geometry}
\usepackage[utf8]{inputenc}
\usepackage[T1]{fontenc}
\usepackage{lmodern}
\usepackage[ngerman]{babel}
\usepackage{csquotes}


\usepackage{pgfgantt} % Für Gantt-Diagramme
\usepackage{hyperref} % Für klickbare Links
\usepackage{geometry} % Für kleinere Seitenränder

\pagestyle{empty}

\title{\Large \thema}
\author{\large \name \\ \large \matnr \\ \large \href{mailto:\email}{\email}}
\date{\large \datum}

\begin{document}

\maketitle

\section{Hintergrund}
Beschreiben Sie hier den Hintergrund des Projekts.

\section{Ziele}
Listen Sie die Ziele Ihres Projekts auf.

\section{Projektbeschreibung}
Beschreiben Sie Ihr Projekt möglichst ausführlich. Nennen Sie gerne Daten, Tools und Bibliotheken die Sie verwenden verwenden und ob Sie Frameworks wie TensorFlow \cite{tensorflow2015} oder PyTorch \cite{pytorch2019} einsetzen werden.

\section{Zeitplan}
Überlegen Sie sich einen Zeitplan für Ihr Projekt. Dieser sollte realistisch sein und genügend Zeit für alle Phasen (im Beispiel unten: Recherche, Prototyp, Feature 1, Feature 2, Dokumentation) einplanen.
Benennen Sie die Phasen Ihres Projekts sinnvoll und geben Sie an, wie lange Sie für jede Phase benötigen.

\noindent % Entfernt Einrückung
\resizebox{\textwidth}{!}{ % Verkleinert das Diagramm
\begin{ganttchart}[
    hgrid,
    vgrid,
    x unit=0.7cm, % Verkleinert die Balkenbreite
    y unit title=0.7cm,
    y unit chart=0.6cm,
    time slot format=simple
  ]{1}{16}
  \gantttitle{Projektzeitraum (16 Wochen)}{16} \\
  \gantttitlelist{1,...,16}{1} \\
  \ganttbar{Recherche}{1}{3} \\
  \ganttbar{Prototyp}{2}{6} \\
  \ganttbar{Feature 1}{4}{10} \\
  \ganttbar{Feature 2}{6}{14} \\
  \ganttbar{Dokumentation}{10}{16}
\end{ganttchart}
}

\section{Literatur}
\begin{thebibliography}{99}

\bibitem{tensorflow2015}
Martín Abadi et al., \emph{TensorFlow: Large-Scale Machine Learning on Heterogeneous Systems}, 2015.  
\url{https://www.tensorflow.org/}

\bibitem{pytorch2019}
Adam Paszke et al., \emph{PyTorch: An Imperative Style, High-Performance Deep Learning Library}, 2019.  
\url{https://pytorch.org/}

\end{thebibliography}

\end{document}
